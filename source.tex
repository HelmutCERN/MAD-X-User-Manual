%%\title{SOURCE}

\section{Source and binaries}
\label{sec:source}

 The source and binaries of \madx can be downloaded from the
 \href{http://cern.ch/madx}{MAD-X web site}, see section release.   

% 
% The .tgz file in there contains all the other files in the directory.
% You find two history files one covering the changes between the latest
% 2 versions and the other gives the full project history up to and including the latest version <h5> madX_full_project.history.</h5>
% The make files Makefile, Makefile_g95 and Makefile_nag work on Linux
% machines and the MAC given the corresponding compiler is available,
% lf95, g95 and nag respectively. There is a Makefile.bat for Windows
% which requires lf95. Then there also Makefile_develop and
% Makefile.prof for compiling on Linux with special strict flags and for
% profiling respectively.
% It is also convenient to go to CERN's <a
% href="http://svnweb.cern.ch/world/wsvn/madx/trunk/madX/#path_trunk_madX_">SVN
% page for MAD-X (world-wide READ access)</a>.
% <p>
% Besides the regular MAD-X versions there are also intermediate one:
% the source code of the latest released version (SVN version might be
% newer) can be found at <a
% href="http://www.cern.ch/Frank.Schmidt/mad-X_newest_src">latest
% released version source directory</a>, while the latest Linux binaries
% of MAD-X (madx (PTC always included!)) can be found in the <a
% href="http://www.cern.ch/Frank.Schmidt/mad-X_bin"> Linux binary
% directory</a>.
% 
% <p><font color="#ff0000">At the same place (courtesy
% Yngve.Inntjore.Levinsen@cern.ch) you will find a RPM of the latest
% version convenient for MAD-X installation on REDHAT like LINUX
% distributions. A DEB package can be provided if somebody volunteers to
% test it.</font></p>
% 
% 
% For Windows you find the executables at the
% location
% <!--- a href="https://mad-x-docs.web.cern.ch/MAD-X-docs/executables.htm">
% <a href="http://www.cern.ch/Frank.Schmidt/windows-binaries/executables.htm">
% MAD-X for Windows</a>
% and for MAC OS X they are at <a
% href="http://www.cern.ch/Frank.Schmidt/MAC-OS">
% MAD-X MAC binary directory</a>.
% <p>
% 
% <address>
% <a href="http://www.cern.ch/Hans.Grote/hansg_sign.html">hansg</a>,
% June 21, 2002
% <a href="http://www.cern.ch/Frank.Schmidt/frs_sign.html">frs</a>,
% August 15, 2003
% </address>
% 

% January 18, 2012 
